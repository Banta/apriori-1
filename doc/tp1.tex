\documentclass[12pt]{article}
\usepackage{sbc-template}
\usepackage{graphicx}
\usepackage[lofdepth,lotdepth]{subfig}
\usepackage{graphics}
\usepackage{amsmath}
\usepackage{wrapfig}
\usepackage{booktabs}
\usepackage{rotating}
\usepackage{times,amsmath,epsfig}
\usepackage{url}
\usepackage{multirow}
 \makeatletter
 \newif\if@restonecol
 \makeatother
 \let\algorithm\relax
 \let\endalgorithm\relax
\usepackage{listings}
\usepackage{float}
\usepackage[lined,algonl,ruled]{algorithm2e}
\usepackage{multirow}
\usepackage[brazil]{babel}
\usepackage[latin1]{inputenc}
\usepackage{enumitem}



% \setlist{nolistsep}

\sloppy

\title{Trabalho Pr�tico 1}

\author{Artur Rodrigues}

\address{Departamento de Ci�ncia da Computa��o \\ Universidade Federal de Minas Gerais (UFMG)
    \email{artur@dcc.ufmg.br}
}

\begin{document}

\maketitle

\section{INTRODU��O}


\section{APRIORI}

\subsection{Complexidade}


\section{PROBLEMA DO ITEM RARO}


\section{BASE DE DADOS}


\section{AVALIA��O EXPERIMENTAL}

\subsection{Procedimentos}

%M�quina onde foi testado

\subsection{An�lise de Par�metros}

\subsection{An�lise da Qualidade da Solu��o}


\section{CONCLUS�O}


\nocite{*}
\bibliographystyle{sbc}
\bibliography{bib}

\end{document}
